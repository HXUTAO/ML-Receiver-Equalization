% (This is included by thesis.tex; you do not latex it by itself.)

\begin{abstract}

% The text of the abstract goes here.  If you need to use a \section
% command you will need to use \section*, \subsection*, etc. so that
% you don't get any numbering.  You probably won't be using any of
% these commands in the abstract anyway.

We apply the techniques from meta-learning and machine learning to the communications domain.  
Specifically, we explore how neural networks can learn to equalize new channel environments without training on them and how neural networks can learn to estimate and correct carrier frequency offset for new rates of rotation without training on them.  
We show that deep neural networks can learn to learn to estimate channel taps for two tap channels.
We also explore how deep recursive neural networks learn to learn to equalize for any given channel.  
We demonstrate that neural networks can learn to learn to estimate and correct carrier frequency offset for new rates of rotation.  Crucially, we do all of this without using backpropagation to re-train the networks for each new set of environmental conditions.

\end{abstract}
